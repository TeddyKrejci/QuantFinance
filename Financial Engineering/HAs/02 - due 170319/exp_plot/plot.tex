\documentclass{article}

\setlength\parindent{0pt}

\title{Financial Engineering - HA 2}
\author{Burzler R., Krejci T., Wittmann M., Zaborskikh E.}
\date{March 17, 2019}

\usepackage{tikz,pgfplots}
\usepackage{amsmath}
\usepackage{amssymb}
\usepackage{epstopdf}
\usepackage{mathtools}
\usepackage{calculator}
\usepackage{a4wide}
\pgfplotsset{compat=1.12}

\usepackage{Sweave}
\begin{document}
\Sconcordance{concordance:plot.tex:plot.Rnw:%
1 17 1 1 0 5 1 1 2 1 0 1 1 1 8 7 0 1 4 2 0 1 4 2 0 1 5 3 0 1 2 7 1 1 5 %
7 0 1 4 3 1}


\maketitle


\begin{Schunk}
\begin{Sinput}
> library(ggplot2)
> lamb_g <- 1/20
> h <- function(x) { 
+   if(x > 20) {
+     a <- 1
+   } else {
+       a <- 0
+       }
+   return(a)
+ }
> f <- function(x) {
+   return(exp(-x)) 
+ }
> g <- function(x) {
+   return(lamb_g*exp(-lamb_g*x)) 
+ }
> fh<-function(x){
+   hx<-sapply(x,h)
+   return(f(x)*hx)
+ }
> seq<-seq(19.5,30,0.001)
> v1<-c(rep(seq,2))
> v2<-c(fh(seq),g(seq)/10^7)
> v3<-factor(c(rep('f(x)h(x)',length(seq)),rep('g(X)',length(seq))))
> dt<-data.frame(v1)
> dt$v2<-v2
> dt$v3<-v3
> names(dt)<-c('x','value','functions')
> ggplot(dt,aes(x=x,y=value,colour=functions))+
+   geom_line()+
+   scale_y_continuous("f(x)h(x)",sec.axis = sec_axis(~ . *10^7, name = "g(x)"))
>           
> 
\end{Sinput}
\end{Schunk}



\end{document}
